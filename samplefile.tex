\documentclass[a4paper,12pt,twoside,openright]{book}
\usepackage[left=2.8cm,top=0.5cm,right=2.8cm,bottom=0.5cm]{geometry}
\renewcommand{\rmdefault}{phv} 
\renewcommand{\sfdefault}{phv} 
\usepackage{amsmath,amssymb}
\usepackage[italian]{babel}
\usepackage{setspace}
\usepackage[utf8]{inputenc}
\begin{document}
\begin{center}
\LARGE{TRIBUNALE ORDINARIO DI}\hrulefill \\
\vspace*{0.5cm}
\normalsize{PROCEDIMENTI SPECIALI SOMMARI}\\
\vspace*{0.5cm}
\LARGE{Nota di iscrizione a ruolo\\
o\\
Nota di accompagnamento}\\
\end{center}
$\text{\rlap{}}\square$
Per ricorrente	
\hspace{1.0cm}		
$\text{\rlap{}}\square$
Per reclamante\vspace{0.5cm}\\
\linespread{1.5}{Si chiede l iscrizione al \textbf{RUOLO GENERALE DEGLI AFFARI CIVILI - PROCEDIMENTI SPECIALI SOMMARI} della seguente causa introdotta con:}\vspace{0.5cm}\\
$\text{\rlap{$\checkmark$}}\square$
(0) Ricorso
\hspace{1.5cm}		
$\text{\rlap{}}\square$
(0) Citazione 
\hspace{1.5cm}	
$\text{\rlap{}}\square$
(7) Reclamo\vspace*{0.5cm}\\
\tiny{Promosso da:}\\
.\dotfill\\
\textbf{C}\tiny{on L'}\textbf{A}\tiny{vv.}  \hspace{3cm}\dotfill\\
\dotfill\\
\textbf{C}\tiny{\textbf{ONTRO}}
\dotfill \\
\textbf{C}\tiny{on L'}\textbf{A}\tiny{vv.}  \dotfill \vspace{1cm}\\
\normalsize$\text{\rlap{}}\square$ Valore della controversia\footnote{Il Valore e determinato ai sensi dell'art.9 Legge 23.12.1999 n. 488 }\dotfill 
Importo del contributo unificato\footnotemark[1]\footnote{Allegare ricevuta di versamento}\dotfill\vspace{1cm} \\
$\text{\rlap{}}\square$ Esenzione dal contributo unificato.
Data di comparizione \hrulefill 
Data di notifica \hrulefill\\
Oggetto e Codice  procedimento sommario [0.10.001..\footnote{indicare oggetto e codice relativo tra quelli elencati in tabella}\\
Oggetto e Codice domanda di merito \hrulefill \footnotemark[3] \\
Oggetto e Codice domanda di merito \hrulefill \footnotemark[3] \\
\framebox{
\begin{minipage}{15cm}

$\text{\rlap{}}\square$ RICORRENTE \hspace{3cm}	 $\text{\rlap{}}\square$ RECLAMANTE\\ NATURA GIURIDICA\footnote{indicare uno dei seguenti codici che identifica la "Natura Giuridica" della parte:
\begin{tabular}{|l|l|l|}
\hline\\
PFI=Persona Fisica & PUM=Pubblico Ministero &	con=Consorzio\\
\hline\\
SOC=Società di capitali &CND=Condominio  &ENP=Ente pubbl o pubb.Amm\\
\hline\\
SOP=Società di persone &EDG=Ente di Gestione &EIS=Ente religioso\\
\hline\\
COP=Cooperativa &ASS=Associazione &PAS=Partito o Sindacato\\
\hline\\
&COM=Comitato &OSE=Stato Est. O org.Intermin.\\
\hline\\
\end{tabular}


}\dotfill ALTRE PARTI N.\dotfill(3)\\
COGNOME NOME O DENOMINAZIONE\dotfill\\
DATA E LUOGO DI NASCITA\dotfill\\
VIA O SEDE\dotfill\\
CODICE FISCALE\dotfill\\
COGNOME E NOME DEL PROCURATORE \dotfill TESSERA N.\dotfill\\
{\raggedleft ORIDNE......................\\\par}
DOMICILIO ELETTO\dotfill\\
COGNOME E NOME DEL PROCURATORE\dotfill\\ TESSERA N.\dotfill\\
{\raggedleft ORIDNE......................\\\par}
\end{minipage}
}

\end{document}
